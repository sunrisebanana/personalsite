%%%%%%%%%%%%%%%%%%%%%%%%%%%%%%%%%%%%%%%%%
% Medium Length Professional CV
% LaTeX Template
% Version 2.0 (8/5/13)
%
% This template has been downloaded from:
% http://www.LaTeXTemplates.com
%
% Original author:
% Trey Hunner (http://www.treyhunner.com/)
%
% Important note:
% This template requires the resume.cls file to be in the same directory as the
% .tex file. The resume.cls file provides the resume style used for structuring the
% document.
%
%%%%%%%%%%%%%%%%%%%%%%%%%%%%%%%%%%%%%%%%%

%----------------------------------------------------------------------------------------
%	PACKAGES AND OTHER DOCUMENT CONFIGURATIONS
%----------------------------------------------------------------------------------------

\documentclass{resume} % Use the custom resume.cls style

\usepackage[left=0.6in,top=0.3in,right=0.6in,bottom=0.3in]{geometry} % Document margins

\name{Jaime Herzog} % Your name
\address{Toronto, Ontario} % Your address
\address{905 - 380 - 5271 \\ jaime.herzog1@gmail.com} % Your phone number and email

\begin{document}

%----------------------------------------------------------------------------------------
%	WORK EXPERIENCE SECTION
%----------------------------------------------------------------------------------------


\begin{rSection}{Experience}
\begin{rSubsection}{Oproma Inc.}{March 2021 - Currently (1.5 Years)}{Full Stack Software Developer}{Ottawa, ON}
\item Design and implement new features for both the frontend and backend of Gardox using Java EE for backend API development, and various Javascript library for the frontend work such as Handlebars and RequireJS
\item Coordinate with team members to meet release deadlines using an Agile methodology, driving sprint reviews on a bi-weekly basis and participating in regular stand-ups
\item Improved the CI/CD structure for Prividox by implementing the Jenkins plugin for Gitlab, enforcing test coverage in merge requests and significantly reducing time spent merging develop to master
\item Provide long term development and customer support for Oproma's legacy product CentralCollab in VB.net by acting as CentralCollab's lead technical point of contact
\item Collaboratively dictate and adhere to coding standards enforced at the project level
\end{rSubsection}

%------------------------------------------------
\begin{rSubsection}{Infosys Ltd.}{November 2020 - March 2021 (5 Months)}{Drupal Developer}{Ottawa, ON}
\item Implemented visual and functional requirements for the CanadaBuys project using Drupal custom module development leveraging PHP
\item Assisted QA team with test plan execution in French language version of project under significant time crunch, ensuring visual and functional features are as specified in visual design documents
\end{rSubsection}

%------------------------------------------------
\begin{rSubsection}{CIBC}{January 2019 - August 2019 (8 Months)}{Intermediate Developer Co-op}{Toronto, ON}
\item Designed, implemented, tested, documented and maintained a data analysis tool, the DSIL Traffic and Error Reporting Web App, to supplement the already existing monitoring applications for the department’s Tier 1 application using Flask, Matplotlib and GitHub
\item Conducted R\&D, gathered user stories, iterated, implemented and documented a desktop notification service for Call Center Agents using a pub/sub model, integrated into legacy software using Visual Basic, C\# and TFS for source control
\end{rSubsection}

%------------------------------------------------

\begin{rSubsection}{2Keys}{May 2017 - December 2017 (8 Months)}{QA Tester and Developer Co-op}{Ottawa, ON}
\item Designed and wrote test plans and test case suites, executed test cases for new features and projects
\item Assessed methods for automating graph extraction from Zenoss monitoring suite, culminating in the AZenoss automation suite, which generated identical graphs generated by the Zenoss frontend, and sorted and output these graphs in the proper format
\end{rSubsection}


%----------------------------------------------------------------------------------------
%	EDUCATION SECTION
%----------------------------------------------------------------------------------------

\begin{rSection}{Education}

{\bf Carleton University} \hfill {September 2015 - May 2020} \\ 
B.S. in Computer Science, Honours COOP, Algorithms stream, Minor in Mathematics  \\
GPA: 9.6/12 (Normalized 3.5)

\end{rSection}

%----------------------------------------------------------------------------------------
%	TECHNICAL STRENGTHS SECTION
%----------------------------------------------------------------------------------------

\begin{rSection}{Skills}

\begin{tabular}{ @{} >{\bfseries}l @{\hspace{6ex}} l }
Languages & Java, Python, C++, Golang,  C\#, SQL \\
Protocols \& APIs & XML, JSON, SOAP, REST \\
Frameworks & Flask, React, Spring Framework \\
Tools & Git, GitHub, Docker, SVN, Kubernetes

\end{tabular}

\end{rSection}



\end{rSection}

%----------------------------------------------------------------------------------------
%	PROJECTS SECTION
%----------------------------------------------------------------------------------------
%\begin{rSection}{Projects}

%\begin{rSubsection}{Brackit}{January 2020 - March 2020}{Developer/Designer}{}
%\item Prepared conceptual proposal detailing user stories and use cases of Brackit, a mobile app designed to run and display tournaments supporting multiple formats, for both tournament organizers and attendees
%\item Designed and implemented the Brackit backend API using Flask, integrating with an existing SQLite database, as well as implemented an algorithm for generating and populating double elimination brackets of any size using an object oriented model
%\item Conducted a demo of the completed product and presented a completed design document for the Brackit app, using UML class diagrams to illustrate the system’s backend and frontend
%\end{rSubsection}

%\end{rSection}

%----------------------------------------------------------------------------------------
%	EXAMPLE SECTION
%----------------------------------------------------------------------------------------

%\begin{rSection}{Section Name}

%Section content\ldots

%\end{rSection}

%----------------------------------------------------------------------------------------

\end{document}
